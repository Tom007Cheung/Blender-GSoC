% Created 2019-04-04 Thu 16:09
% Intended LaTeX compiler: pdflatex
\documentclass[11pt]{article}
\usepackage[utf8]{inputenc}
\usepackage[T1]{fontenc}
\usepackage{graphicx}
\usepackage{grffile}
\usepackage{longtable}
\usepackage{wrapfig}
\usepackage{rotating}
\usepackage[normalem]{ulem}
\usepackage{amsmath}
\usepackage{textcomp}
\usepackage{amssymb}
\usepackage{capt-of}
\usepackage{hyperref}
\author{Yadong Cheung}
\date{\textit{<2019-03-31 Sun 21:31>}}
\title{Add more intelligent shortcut keys for outliner <draft>\\\medskip
\large Proposal.pdf would be written in \LaTeX}
\hypersetup{
 pdfauthor={Yadong Cheung},
 pdftitle={Add more intelligent shortcut keys for outliner <draft>},
 pdfkeywords={GSoC Blender},
 pdfsubject={Apply for Blender of GSoC 2019},
 pdfcreator={<a href="https://www.gnu.org/software/emacs/">Emacs</a> 26.1 (<a href="https://orgmode.org">Org</a> mode 9.2)}, 
 pdflang={English}}
\begin{document}

\maketitle


\section*{Name}
\label{sec:org9a11a28}
Yadong Cheung 张亚栋

\section*{Contact}
\label{sec:orge37a461}
\textbf{Email:}

zhyd007@gmail.com

10154508169@stu.ecnu.edu.cn

\textbf{IRC nick:} Cheung

\section*{Synopsis}
\label{sec:org8f5703a}
  For artists, keyboard shortcuts may more convenient than buttons clicking, and most of all features can be implement, such
 as "open files \emph{<Ctrl+O>}", but there are some important features we need shortcuts, for example,to select all items 
in-between, we  must click all one. 
The project is designed to add more intelligent shortcuts to improve outliner.

\section*{Benefits}
\label{sec:org7bc30ae}
More user-friendly and powerful outliner.

\section*{Deliverables}
\label{sec:orge87d1f9}
\subsection*{Improvemt User Experience of Outliner.}
\label{sec:org5d8b60c}
\begin{itemize}
\item Enter \emph{<Del>} to delete object.
\item Click the object you want to choose, press \emph{<Shift>} and click the last one you want, you will choose all between them.
\item Click the first item you want to choose, press \emph{<Ctrl>} and click other ones, you can choose them ignore the other between them.
\item Add a component "Synced Selection".
\end{itemize}
\section*{Project Details [3/6]}
\label{sec:org5ae2747}
 \textbf{code layout}: \emph{space\_outliner --> interface --> object --> windowmanager} \\
 The major code is in \emph{source/blender/editors/} \textbf{space\_outliner} \&\& \textbf{space\_view3d}
path.\\
\begin{itemize}
\item[{$\boxtimes$}] Shortcut keys for operations.\\
\textbf{wm\_keymap.c} defines keymap and includes \textbf{wm\_event\_types.h}, which contains \emph{keyboard codes.}
\item[{$\boxtimes$}] Sync selection between 3D viewport and outliner.\\
We only need one way to show selections - the highlighted row.
\item[{$\boxtimes$}] Shift-select to select a range of items.\\
It is related to \textbf{tree's traversal} problem, and shortcut keys binding.
\item[{$\square$}] Box selection by simple click and drag.\\
\item[{$\square$}] Arrow key navigation.
\item[{$\square$}] More consistent and powerful right-click menus.
\end{itemize}
\section*{Project Schedule}
\label{sec:orgc8efe29}
I could finish this project in 12 weeks, and about 1 item per 2 weeks.(more details\ldots{})

\section*{Bio}
\label{sec:orga40ab8d}
  My name is Yadong Cheung, I live in China.
Currently studying at East China Normal University(ECNU).
I major in preschool education, UI design is my hobby, and I love 3D modules especially in animals.
The first time that I know CG is the interdisciplinary course \emph{The Film and Television Direct} 
I studied, which needed pecial effects modeling. HTML and CSS is based on web and it's designed for
front-end engineer, maybe they could be helpful of outliner designing, I also started learning \TeX, the perfect
typesetting system, last year. 

\subsection*{Programming experience}
\label{sec:org01b8281}
I like \textbf{C} language, and although I have no experience in products but I read the two books which are konwn as
 \textbf{The C Programming Language} \emph{2e} and
\textbf{C Primer Plus} \emph{6e}. I used to programming in XCode and Eclipse.
I learned \textbf{Python} from edx, which course is
 \href{https://www.edx.org/course/introduction-to-computer-science-and-programming-using-python-0}{Introduction to Computer Science and Programming Using Python}.\\
 \vfill
Edited by \href{https://www.gnu.org/software/emacs/}{Emacs}26.1 \href{https://orgmode.org}{Org  mode} 9.2
\end{document}
